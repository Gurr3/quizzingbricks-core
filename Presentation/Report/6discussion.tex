\subsection{Development Issues}
When we planned the project we decided that with so many members we could create both an Android and an iOS client to prove how cross platforming would work. This meant setting two people to work on the android client and one on the iOS client thinking that the clients would be done in good time before the end of the course. Unfortunately we did not consider the steep learning curve of creating a full application with little previous experience. This meant that the three people set to work on the clients were pretty much locked in to clients with too much work spent to scrap them when we realised that they would take the full extent of the course to complete. So in hindsight we would probably have gone with only one of the mobile clients. 

Another problem we had was the difficulty in getting something to work as early as possible without being completely done with it. Most features we implemented were unusable until they were finished since we set to make them as good as we wanted them to be for the finished product from the beginning. This is probably where most of us have learned most from our mistake since it is so much more useful to have a bare bone working prototype to share with the team than nothing at all until you have a finished product. Especially since you are unlikely to have achieved a finished product without testing it with the team. 

In both the android and the iOS client the focus has been to get all features working and not in trying to make a beautiful designed interface shared between devices. So for each client we've stuck to the standard way of making a simple application. This has lead to some minor differences between the mobile clients but we believe that trying to for example recreate the iOS layout in the android phone or vice versa would only confuse the user more since an android or iOS user expects an application to work in a certain way when using an android or iOS device. 
\subsection{Future Work}
The most prominent work would be to actually launch the application. What is stopping us today is that we have no questions for our quiz game, for this purpose we would need to buy a service providing this.
For the application designs on web and the mobile clients more time could be spent on graphical design to make the clients look better, also reducing the clicks needed to do things inside the application by using "smart" patterns or additional settings so you for example, automatically accept invite from a player etc.
For the game itself, one can think of developing additional game modes that you can choose to play.
